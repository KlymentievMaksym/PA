\documentclass{article}
\usepackage{graphicx}
% \usepackage[paperheight=16cm,paperwidth=12cm,textwidth=10cm]{geometry}
\usepackage{lipsum}
\usepackage[T2A]{fontenc}
% \usepackage[koi8-r]{inputenc}
\usepackage[utf8]{inputenc}

\graphicspath{ {./Results/} }

\title{Quick Sort}
\author{Dadmo}
\date{\today}

\begin{document}
\maketitle

\tableofcontents 
\section{Quick Sort}
\textbf{QuickSort} --- клас, який має реалiзованi 4 (16) варiантів алгоритму сортування.
\newline
\textbf{Реалiзовано 4 варiанти розбиття}: 
\begin{enumerate}
    \item Розбиття Ломуто
    \item Розбиття Гоара
    \item Розбиття Дейкстри
    \item Двоелементне Розбиття
\end{enumerate}
\textbf{Для кожно розбиття реалiзовано 4 варiанти вибору опорного елемента}:
\begin{enumerate}
    \item Останній елемент
    \item Випадковий елемент
    \item Медіана першого, середнього та останього елемента
    \item Медіана трьох випадкових елементів
\end{enumerate}
\end{document}